% Exam Template for UMTYMP and Math Department courses
%
% Using Philip Hirschhorn's exam.cls: http://www-math.mit.edu/~psh/#ExamCls
%
% run pdflatex on a finished exam at least three times to do the grading table on front page.
%
%%%%%%%%%%%%%%%%%%%%%%%%%%%%%%%%%%%%%%%%%%%%%%%%%%%%%%%%%%%%%%%%%%%%%%%%%%%%%%%%%%%%%%%%%%%%%%

% These lines can probably stay unchanged, although you can remove the last
% two packages if you're not making pictures with tikz.
\documentclass[11pt]{exam}
\RequirePackage{amssymb, amsfonts, amsmath, latexsym, verbatim, xspace, setspace, blkarray, multirow, array}
\RequirePackage{tikz, pgflibraryplotmarks, pgfplotstable}
\RequirePackage{booktabs,pgfplots,pgfplotstable}

% By default LaTeX uses large margins.  This doesn't work well on exams; problems
% end up in the "middle" of the page, reducing the amount of space for students
% to work on them.
\usepackage[margin=1in]{geometry}


% Here's where you edit the Class, Exam, Date, etc.
\newcommand{\class}{Simulaci\'on}
\newcommand{\code}{}
\newcommand{\term}{Primavera 2018}
\newcommand{\examnum}{Examen Recuperativo}
\newcommand{\examdate}{18/12/18}
\newcommand{\timelimit}{90 Minutos}

% For an exam, single spacing is most appropriate
\singlespacing
% \onehalfspacing
% \doublespacing

% For an exam, we generally want to turn off paragraph indentation
\parindent 0ex

%
\begin{document} 

% These commands set up the running header on the top of the exam pages
\pagestyle{head}
\firstpageheader{}{}{}
\runningheader{\class}{\examnum\ - P\'agina \thepage\ de \numpages}{\examdate}
\runningheadrule

\begin{flushright}
\begin{tabular}{p{2.8in} r l}
\textbf{\class} & \textbf{Nombre:} & \makebox[2in]{\hrulefill}\\
\textbf{\code} && \textbf{C\'odigo de honor:} \\
\textbf{\term} && No he dado ni recibido\\
\textbf{\examnum} && ayuda durante este certamen\\
\textbf{\examdate} && \\
\textbf{Tiempo l\'imite: \timelimit} & \textbf{Firma} & \makebox[2in]{\hrulefill}
\end{tabular}\\
\end{flushright}
\rule[1ex]{\textwidth}{.1pt}


Este certamen contiene \numpages\ p\'aginas (incluyendo esta cubierta) y 
\numquestions\ preguntas.  Cerciorece que su copia contiene todas las p\'aginas.  Ponga su iniciales arriba de cada p\'agina en el caso de que separe las hojas y estas se puedan perder.\\

Usted \textbf{PUEDE} utilizar una hoja A4 escrita en una de sus carillas para el certamen.\\

Se requiere que muestre su trabajo para cada problema en este certamen.  Las siguientes reglas aplican:\\

\begin{minipage}[t]{3.7in}
\vspace{0pt}
\begin{itemize}

\item \textbf{Organize su trabajo}, de forma razonablemente ordenada, en el espacio entregado. Trabajo desorganizado dif\'icil de evaluar recibir\'a poco o nada de puntaje (independiente de su exactitud). 

\item \textbf{Respuestas misteriosas o sin fundamentos no recibir\'an puntaje}.  Una respuesta correcta, sin soporte de calculos, explicaci\'on, o trabajo algebraico \textbf{NO} recibir\'a puntaje; una respuesta incorrecta que sea el resultado de calculos intermedios correctos podr\'ia recibir puntaje parcial.

\item Si necesita mas espacio, use el reverso de la p\'agina; indique claramente cuando haga esto.
\end{itemize}

No escriba en la tabla a la derecha.
\end{minipage}
\hfill
\begin{minipage}[t]{2.3in}
\vspace{0pt}
%\cellwidth{3em}
\gradetablestretch{2}
\vqword{Problem}
\addpoints % required here by exam.cls, even though questions haven't started yet.	
\gradetable[v]%[pages]  % Use [pages] to have grading table by page instead of question

\end{minipage}
\newpage % End of cover page

%%%%%%%%%%%%%%%%%%%%%%%%%%%%%%%%%%%%%%%%%%%%%%%%%%%%%%%%%%%%%%%%%%%%%%%%%%%%%%%%%%%%%
%
% See http://www-math.mit.edu/~psh/#ExamCls for full documentation, but the questions
% below give an idea of how to write questions [with parts] and have the points
% tracked automatically on the cover page.
%
%
%%%%%%%%%%%%%%%%%%%%%%%%%%%%%%%%%%%%%%%%%%%%%%%%%%%%%%%%%%%%%%%%%%%%%%%%%%%%%%%%%%%%%
\begin{questions}
\section*{Probability theory - Certamen 1}
% Basic question
\addpoints
\question Una variable aleatoria $Y$ tiene la siguiente funci\'on de densidad:

 \[f_Y(x)=
\begin{cases}
0&\text{for $x < 0$}\\
\frac{3}{16}x^2+\frac{1}{4}& \text{for $0\leq x < 2$}\\
0&\text{for $2 < x$}\\
\end{cases}
\]

\begin{parts}
\part[4] ?`Cu\'al es el valor esperado de Y?
%\fillwithlines{1.5 in}
\fillwithlines{3.5 in}
\part[6] ?`Cu\'al es la funci\'on de densidad acumulada de Y? ?`Es m\'as probable obtener un valor cercano a $1/2$ o a $3/2$?%\fillwithlines{3 in}
\fillwithlines{3 in}
\end{parts}

\newpage
\section*{Validaci\'on de distribuci\'on - Certamen 1}

\question Usted quiere modelar el n\'umero diario de clientes que van a un restaurant. El due\~no del local entrega una aproximaci\'on te\'orica de como los clientes se distribuyen durante la semana:

\begin{table}[!htbp]
\centering
\begin{tabular}{lcccccc}
\toprule
D\'ia&L&M&M&J&V&S\\
\midrule
Porcentaje (\%) & 15 & 10 & 15 & 20 & 25 & 15\\
\bottomrule
\end{tabular}
\end{table}

Antes de utilizar esta distribuci\'on te\'orica usted decide validarla, por lo cual va a recolectar datos y obtiene la siguiente tabla:

\begin{table}[!htbp]
\centering
\begin{tabular}{lccccccc}
\toprule
D\'ia&L&M&M&J&V&S&Total\\
\midrule
N\'umero de Clientes & 26 & 18 & 34 & 45 & 50 & 27 & 200\\
\bottomrule
\end{tabular}
\end{table}

\begin{parts}
\part[15] Comprobar si la distribuci\'on te\'orica sirve para modelar el flujo de clientes (Use Chi-cuadrado de tabla de 11.45) 
\fillwithlines{5 in}
\end{parts}


\section*{Likelihood y moment matching - Certamen 1}
%\iffalse
\question Suponga que $X$ es una variable aleatoria discreta con funci\'on de probabilidad de masa.

\begin{table}[!htbp]
\centering
\begin{tabular}{lcccc}
\toprule
X&0&1&2&3\\
\midrule
$P(X=x)$ & $2\theta/3$ & $\theta/3$& $2(1-\theta)/3$ & $(1-\theta)/3$\\
\bottomrule
\end{tabular}
\end{table}
\begin{parts}
\part[15] Se han obtenido diez muestras de la distribuci\'on (3,0,2,1,3,2,1,0,2,1), y se le pide determinar el par\'ametro $\hat{\theta}$.
\fillwithlines{6 in}
\end{parts}

\section*{Procesos Especiales - Certamen 2}

\question Usted necesita determinar los tiempos de llegada de un proceso de intensidad variable en el tiempo. La siguiente informaci\'on ha sido obtenida:

 \[\lambda(x)=
\begin{cases}
2.5&\text{for $x\in[0,10[$}\\
3&\text{for $x\in[10,20[$}\\
4.2&\text{for $x\in[20,30]$}\\
\end{cases}
\]

\begin{parts}
\part[7] Determine la distribuci\'on inversa acumulada.
\fillwithlines{4.2 in}
\part[3] Utilizando los siguientes n\'umeros aleatorios: 0.375, 0.063, y 0.0.5  determine el tiempo de las primeras tres llegadas al sistema.
\fillwithlines{3 in}
\end{parts}
\newpage
\section*{Variables Aleatorias - Certamen 2}
\question Utilizando los n\'umeros aleatorios 0.6754, 0.8602.

\begin{parts}
\part[5] Genere una variable aleatoria Weibull(a,b) con $a=5, b=3$ (Ayuda: $F(x)=1-\text{exp}[-(x/a)^b]$, para $x>0,~ 0$ de otro modo) usando el primer n\'umero aleatorio.
\fillwithlines{3.4 in}
\part[2] Genere una variable aleatoria exponencial con $\lambda = 5$ usando el segundo n\'umero aleatorio
\fillwithlines{3.6 in}
\newpage
\part[3] Utilizando una mezcla de las dos distribuciones anteriores con pesos $p_1 = 0.3$ y $p_2 = 1-p_1$, y con n\'umeros aleatorios 0.1453, 0.8763 genere una variable aleatoria mixta.
\fillwithlines{2 in}
\end{parts}

\end{questions}
\end{document}