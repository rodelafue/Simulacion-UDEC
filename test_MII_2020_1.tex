% Exam Template for UMTYMP and Math Department courses
%
% Using Philip Hirschhorn's exam.cls: http://www-math.mit.edu/~psh/#ExamCls
%
% run pdflatex on a finished exam at least three times to do the grading table on front page.
%
%%%%%%%%%%%%%%%%%%%%%%%%%%%%%%%%%%%%%%%%%%%%%%%%%%%%%%%%%%%%%%%%%%%%%%%%%%%%%%%%%%%%%%%%%%%%%%

% These lines can probably stay unchanged, although you can remove the last
% two packages if you're not making pictures with tikz.
\documentclass[11pt]{exam}
\RequirePackage{amssymb, amsfonts, amsmath, latexsym, verbatim, xspace, setspace, blkarray, multirow, array}
\RequirePackage{tikz, pgflibraryplotmarks, pgfplotstable}
\RequirePackage{booktabs,pgfplots,pgfplotstable}

% By default LaTeX uses large margins.  This doesn't work well on exams; problems
% end up in the "middle" of the page, reducing the amount of space for students
% to work on them.
\usepackage[margin=1in]{geometry}


% Here's where you edit the Class, Exam, Date, etc.
\newcommand{\class}{Simulaci\'on}
\newcommand{\code}{}
\newcommand{\term}{Oto\~no 2020}
\newcommand{\examnum}{Certamen 1}
\newcommand{\examdate}{05/03/20}
\newcommand{\timelimit}{90 Minutos}

% For an exam, single spacing is most appropriate
\singlespacing
% \onehalfspacing
% \doublespacing

% For an exam, we generally want to turn off paragraph indentation
\parindent 0ex

%
\begin{document} 

% These commands set up the running header on the top of the exam pages
\pagestyle{head}
\firstpageheader{}{}{}
\runningheader{\class}{\examnum\ - P\'agina \thepage\ de \numpages}{\examdate}
\runningheadrule

\begin{flushright}
\begin{tabular}{p{2.8in} r l}
\textbf{\class} & \textbf{Nombre:} & \makebox[2in]{\hrulefill}\\
\textbf{\code} && \textbf{C\'odigo de honor:} \\
\textbf{\term} && No he dado ni recibido\\
\textbf{\examnum} && ayuda durante este certamen\\
\textbf{\examdate} && \\
\textbf{Tiempo l\'imite: \timelimit} & \textbf{Firma} & \makebox[2in]{\hrulefill}
\end{tabular}\\
\end{flushright}
\rule[1ex]{\textwidth}{.1pt}


Este certamen contiene \numpages\ p\'aginas (incluyendo esta cubierta) y 
\numquestions\ preguntas.  Cerciorece que su copia contiene todas las p\'aginas.  Ponga su iniciales arriba de cada p\'agina en el caso de que separe las hojas y estas se puedan perder.\\

Usted \textbf{PUEDE} utilizar una hoja A4 escrita en una de sus carillas para el certamen.\\

Se requiere que muestre su trabajo para cada problema en este certamen.  Las siguientes reglas aplican:\\

\begin{minipage}[t]{3.7in}
\vspace{0pt}
\begin{itemize}

\item \textbf{Organize su trabajo}, de forma razonablemente ordenada, en el espacio entregado. Trabajo desorganizado dif\'icil de evaluar recibir\'a poco o nada de puntaje (independiente de su exactitud). 

\item \textbf{Respuestas misteriosas o sin fundamentos no recibir\'an puntaje}.  Una respuesta correcta, sin soporte de calculos, explicaci\'on, o trabajo algebraico \textbf{NO} recibir\'a puntaje; una respuesta incorrecta que sea el resultado de calculos intermedios correctos podr\'ia recibir puntaje parcial.

\item Si necesita mas espacio, use el reverso de la p\'agina; indique claramente cuando haga esto.
\end{itemize}

No escriba en la tabla a la derecha.
\end{minipage}
\hfill
\begin{minipage}[t]{2.3in}
\vspace{0pt}
%\cellwidth{3em}
\gradetablestretch{2}
\vqword{Problem}
\addpoints % required here by exam.cls, even though questions haven't started yet.	
\gradetable[v]%[pages]  % Use [pages] to have grading table by page instead of question

\end{minipage}
\newpage % End of cover page

%%%%%%%%%%%%%%%%%%%%%%%%%%%%%%%%%%%%%%%%%%%%%%%%%%%%%%%%%%%%%%%%%%%%%%%%%%%%%%%%%%%%%
%
% See http://www-math.mit.edu/~psh/#ExamCls for full documentation, but the questions
% below give an idea of how to write questions [with parts] and have the points
% tracked automatically on the cover page.
%
%
%%%%%%%%%%%%%%%%%%%%%%%%%%%%%%%%%%%%%%%%%%%%%%%%%%%%%%%%%%%%%%%%%%%%%%%%%%%%%%%%%%%%%
\begin{questions}
\section*{Probabilidades}
% Basic question
\addpoints
\question Una m\'aquina produce componentes que son ya sea aceptables o defectuosos. Despu\'es de observar 200 pares de componentes, la siguiente informaci\'on fue recolectada:


 \begin{table}[h!]
	\centering
    \setlength{\extrarowheight}{2pt}
    \begin{tabular}{cc|c|c|}
      & \multicolumn{1}{c}{} & \multicolumn{2}{c}{Segundo}\\
      & \multicolumn{1}{c}{} & \multicolumn{1}{c}{Aceptable}  & \multicolumn{1}{c}{Defectuoso} \\\cline{3-4}
      \multirow{2}*{Primero}  & Aceptable & 120 & 20 \\\cline{3-4}
      & Defectuoso & 40 & 20 \\\cline{3-4}
    \end{tabular}
  \end{table}


Dej\'emos que $X_1$ represente el primer componente y $X_2$ el segundo. Los trabajadores asignaron el valor 0 a un componente aceptable y el de 1 a un componente defectuoso. Conteste las siguientes preguntas:
\begin{parts}
\part[4] Estime la funci\'on de probabilidad de masa conjunta para estas dos variables aleatorias.
%\fillwithlines{1.5 in}
\fillwithlines{2 in}
\part[6] ?`Cu\'al es la probabilidad que el segundo componente sea defectuoso si el primer componente no es defectuoso? ?`Cu\'al es la probabilidad que el segundo componente sea defectuoso? ?`Son $X_1$ y $X_2$ independientes?%\fillwithlines{3 in}
\fillwithlines{2 in}
\end{parts}
\newpage

%############################################
%################## METRICAS ################

\section*{C\'alculo de m\'etricas en simulaci\'on}
\question Usted decide testear si su entendimiento sobre como opera internamente una simulaci\'on es el adecuado. Para ello va a un restaurant y comienza a observar su operaci\'on y va registrando lo que ve en una hoja, tal como se observa en la Tabla \ref{tab:1} 
\begin{table} \label{tab:1}
\center
\begin{tabular}{llll}
\hline
\textbf{Hora} & \textbf{Evento} & \textbf{Trabajadores} & \textbf{Proceso}                                                  \\ \hline
8:00:00               & Abrir tienda                & 1                     & Abre tienda - Servidor 1 desocupado                               \\
8:12:26            & Llega Cliente 1                & 1                     & Llega Cliente 1 - Servidor 1 ocupado                              \\
8:21:36            & Llega Cliente 2                & 1                     & Llega Cliente 2 - Espera en Cola                            \\
8:25:30          & Salida Cliente 1                & 1                     & Cliente 1 sale - Cliente 2 en atenci\'on                              \\
8:27:00           & Llega Cliente 3                & 1                     & Llega Cliente 3 - Espera en cola                                  \\
8:30:00           & Fin observaci\'on                & 1                     &Fin periodo                                 \\ \hline
\end{tabular}
\caption{Mi Hoja de Registros}
\end{table}

\begin{parts}
\part[3] ?`Cu\'al es el tiempo promedio en el sistema?
\fillwithlines{2 in}
\part[8] ?`Cu\'al es n\'umero promedio en el sistema?
\fillwithlines{2 in}
\part[4] ?`Cu\'al es la utilizaci\'on del servidor?
\fillwithlines{2 in}
\end{parts}

%########################################################
%################## MODELO SIMULACION ###################

\section*{Modelo de Simulaci\'on}

\question Usted dispone del modelo de simulaci\'on mas sencillo que consite de un SOURCE, un SERVER, y un SINK. Se le pide que le haga las siguientes modificaciones:
\begin{parts}
\part[5] El SERVER tiene capacidad diferenciada de trabajo dentro del d\'ia. Capacidad 1 dentro de las primeras 3 horas, capacidad 2 las siguientes 2 horas, luego un break de una hora, y finalmente capacidad 1 durante las \'ultimas 3 horas del d\'ia. Adem\'as ha observado que el tiempo m\'inimo de atenci\'on es de 3 minutos, el m\'as probable es de 5, mientras que el m\'aximo es 7.
\part[5] Tambi\'en se ha dado cuenta que la tasa de llegada del SOURCE cambia durante el d\'ia, y se comporta de la siguiente manera: 10 personas por hora dentro de las primeras 3 horas, 15 personas por hora las siguientes 2 horas, luego un break de una hora, y finalmente 12 personas por hora durante las \'ultimas 3 horas del d\'ia.
\part[5] Determine el n\'umero promedio de personas en el sistema entre la segunda y cuarta hora del d\'ia.
\end{parts}


%########################################################
%################## COMPARACION SISTEMAS ################

\section*{Comparaci\'on de sistemas alternativos}
\question Usted desea saber si dos configuraciones de su sistema son estad\'isticamente diferentes y ha recolectado la siguiente informaci\'on.
 \begin{table}[h!]
	\centering
    \setlength{\extrarowheight}{2pt}
    \begin{tabular}{ccccccccccc}
		\toprule
		Experiment & Rep 1& Rep 2& Rep 3& Rep 4& Rep 5\\
		\midrule
			+ + + & 13.4 & 14.78 & 18.52 & 9.56 & 9.92 \\
			+ - + & 7.45 & 9.67 & 8.92 & 5.32 & 6.65 \\
		\bottomrule
    \end{tabular}
  \end{table}
\begin{parts}
\newpage
\part[6] Determine si las configuraciones producen resultados estad\'isticamente diferentes. Asuma que se utilizaron n\'umeros aleatorios comunes.
\fillwithlines{2.8 in}
\fillwithlines{2.8 in}
\part[4] Usted quiere reducir el ancho medio de primer experimento a un 20\% de su valor inical. ?`Cu\'antas muestras adicionales necesita?
\fillwithlines{2 in}
\end{parts}

%########################################################
%################## GENERALIDADES #######################

\section*{Generalidades}
\question Responda las siguientes preguntas:
\begin{parts}
\part[6]?` Qu\'e diferencia existe entre una r\'eplica, un escenario y un experimento? Explique. 
\fillwithlines{3 in}
\part[4] ?` Cu\'ando es conveniente utilizar n\'umeros aleatorios comunes (common random numbers)?. 
\fillwithlines{3 in}
\end{questions}
\end{document}