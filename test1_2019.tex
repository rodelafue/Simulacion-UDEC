% Exam Template for UMTYMP and Math Department courses
%
% Using Philip Hirschhorn's exam.cls: http://www-math.mit.edu/~psh/#ExamCls
%
% run pdflatex on a finished exam at least three times to do the grading table on front page.
%
%%%%%%%%%%%%%%%%%%%%%%%%%%%%%%%%%%%%%%%%%%%%%%%%%%%%%%%%%%%%%%%%%%%%%%%%%%%%%%%%%%%%%%%%%%%%%%

% These lines can probably stay unchanged, although you can remove the last
% two packages if you're not making pictures with tikz.
\documentclass[11pt]{exam}
\RequirePackage{amssymb, amsfonts, amsmath, latexsym, verbatim, xspace, setspace, blkarray, multirow, array}
\RequirePackage{tikz, pgflibraryplotmarks, pgfplotstable}
\RequirePackage{booktabs,pgfplots,pgfplotstable}
\usepackage{float}
% By default LaTeX uses large margins.  This doesn't work well on exams; problems
% end up in the "middle" of the page, reducing the amount of space for students
% to work on them.
\usepackage[margin=1in]{geometry}


% Here's where you edit the Class, Exam, Date, etc.
\newcommand{\class}{Simulaci\'on}
\newcommand{\code}{}
\newcommand{\term}{Primavera 2019}
\newcommand{\examnum}{Certamen 1}
\newcommand{\examdate}{04/10/19}
\newcommand{\timelimit}{90 Minutos}

% For an exam, single spacing is most appropriate
\singlespacing
% \onehalfspacing
% \doublespacing

% For an exam, we generally want to turn off paragraph indentation
\parindent 0ex

%
\begin{document} 

% These commands set up the running header on the top of the exam pages
\pagestyle{head}
\firstpageheader{}{}{}
\runningheader{\class}{\examnum\ - P\'agina \thepage\ de \numpages}{\examdate}
\runningheadrule

\begin{flushright}
\begin{tabular}{p{2.8in} r l}
\textbf{\class} & \textbf{Nombre:} & \makebox[2in]{\hrulefill}\\
\textbf{\code} && \textbf{C\'odigo de honor:} \\
\textbf{\term} && No he dado ni recibido\\
\textbf{\examnum} && ayuda durante este certamen\\
\textbf{\examdate} && \\
\textbf{Tiempo l\'imite: \timelimit} & \textbf{Firma} & \makebox[2in]{\hrulefill}
\end{tabular}\\
\end{flushright}
\rule[1ex]{\textwidth}{.1pt}


Este certamen contiene \numpages\ p\'aginas (incluyendo esta cubierta) y 
\numquestions\ preguntas.  Cerciorece que su copia contiene todas las p\'aginas.  Ponga su iniciales arriba de cada p\'agina en el caso de que separe las hojas y estas se puedan perder.\\

Usted \textbf{PUEDE} utilizar una hoja A4 escrita en una de sus carillas para el certamen.\\

Se requiere que muestre su trabajo para cada problema en este certamen.  Las siguientes reglas aplican:\\

\begin{minipage}[t]{3.7in}
\vspace{0pt}
\begin{itemize}

\item \textbf{Organize su trabajo}, de forma razonablemente ordenada, en el espacio entregado. Trabajo desorganizado dif\'icil de evaluar recibir\'a poco o nada de puntaje (independiente de su exactitud). 

\item \textbf{Respuestas misteriosas o sin fundamentos no recibir\'an puntaje}.  Una respuesta correcta, sin soporte de calculos, explicaci\'on, o trabajo algebraico \textbf{NO} recibir\'a puntaje; una respuesta incorrecta que sea el resultado de calculos intermedios correctos podr\'ia recibir puntaje parcial.

\item Si necesita mas espacio, use el reverso de la p\'agina; indique claramente cuando haga esto.
\end{itemize}

No escriba en la tabla a la derecha.
\end{minipage}
\hfill
\begin{minipage}[t]{2.3in}
\vspace{0pt}
%\cellwidth{3em}
\gradetablestretch{2}
\vqword{Problem}
\addpoints % required here by exam.cls, even though questions haven't started yet.	
\gradetable[v]%[pages]  % Use [pages] to have grading table by page instead of question

\end{minipage}
\newpage % End of cover page

%%%%%%%%%%%%%%%%%%%%%%%%%%%%%%%%%%%%%%%%%%%%%%%%%%%%%%%%%%%%%%%%%%%%%%%%%%%%%%%%%%%%%
%
% See http://www-math.mit.edu/~psh/#ExamCls for full documentation, but the questions
% below give an idea of how to write questions [with parts] and have the points
% tracked automatically on the cover page.
%
%
%%%%%%%%%%%%%%%%%%%%%%%%%%%%%%%%%%%%%%%%%%%%%%%%%%%%%%%%%%%%%%%%%%%%%%%%%%%%%%%%%%%%%
\begin{questions}
\section*{Probability theory}
% Basic question
\addpoints
\question[5] Sea $X$ una variable aleatoria discreta, con soporte en $R_X = \{0,1,2,3\}$. Adem\'as su funci\'on de probabilidad de masa $p_X(x)$ esta dada por:

\begin{equation}
p_X(x)=\left\{\begin{matrix}
\begin{pmatrix}
3\\ 
x
\end{pmatrix} \left (\frac{1}{4}  \right )^x \left (\frac{3}{4}  \right )^{3-x}&  , x \in R_X\\ 
0 & , x \notin R_X
\end{matrix}\right.
\end{equation}
Donde,

\begin{equation}
\binom{3}{x}=\frac{3!}{x!\left ( 3-x \right )!}
\end{equation}

\begin{parts}
\part Calcule la probabilidad $P(X<3)$.
\fillwithlines{4 in}

\end{parts}


\question Se lanzan tres monedas al aire y la posibilidad de los resultados de cada una puede ser cara $(H)$ o sello $(T)$. La primera moneda no esta cargada pero la segunda lo esta en $0.6$ para $H$, y la tercera en 0.9 para $H$. El experimento se realiza como parte de un juego en el cual por cada cara $(H)$ se pierden \$1000 y por cada sello se ganan \$1000.

\begin{parts}
\part[8] Defina el espacio muestral del experimento y sus probabilidades.
\fillwithlines{4 in}
\part[4] Defina la variable aleatoria que caracteriza las ganancias del juego.
\fillwithlines{4 in}
\part[3] Grafique la probabilidad de masa asociada a las ganancias.
\fillwithlines{2 in}
\end{parts}

\question[10] Sea $X$ una variable aleatoria continua de dimensiones $2 \times 1$ y defina sus componentes como $X_1$ y $X_2$. Adem\'as, defina el soporte de $X$ como $R_X=[0,2]\times[0,3]$ (el conjunto de vectores de dimension $2\times 1$ tal que el primer componente pertenece al intervalo $[0,2]$ y el segundo al $[0,3]$). La funci\'on de densidad conjunta de $X$ es:

\begin{equation}
f_X(x)=\left\{\begin{matrix}
1/6 &, \text{si}~~x \in R_X \\ 
0 & ~~~~~,\text{de otra forma}. 
\end{matrix}\right.
\end{equation}

\begin{parts}
\part Calcule $P(1\leq X_1 \leq 3, -1\leq X_2 \leq 1)$
\fillwithlines{4 in}
\end{parts}

\newpage
\section*{C\'alculo de m\'etricas en simulaci\'on}
\question Usted decide testear si su entendimiento sobre como opera internamente una simulaci\'on es el adecuado. Para ello va a un restaurant y comienza a observar su operaci\'on y registra todo lo que observa en una hoja, tal como se observa en la Tabla \ref{tab:1}.
\begin{table}[H] \label{tab:1}
\begin{tabular}{llll}
\hline
\textbf{Tiempo} & \textbf{Cliente} & \textbf{Trabajadores} & \textbf{Proceso}                                                  \\ \hline
0               & -                & 1                     & Abre tienda - Heladero 1 desocupado                               \\
2:56            & 1                & 1                     & Llega Cliente 1 - Heladero 1 ocupado                              \\
7:21            & 1                & 1                     & Sale Cliente 1 - Heladero 1 desocupado                            \\
10:50           & 2                & 1                     & Llega Cliente 2 - Heladero 1 ocupado                              \\
11:30           & 3                & 1                     & Llega Cliente 3 - Espera en cola                                  \\
15:07           & 4                & 1                     & Llega Cliente 4 -  Espera en cola                                 \\
16:46           & -                & 2                     & Llega Heladero 2 - Se prepara para trabajar                                              \\
17:10           & 3                & 2                     & Cliente 3 comienza servicio - Heladero 2 ocupado                  \\
17:20           & 2 - 4            & 2                     & Sale Cliente 2 - Cliente 4 comienza servicio \\
18:32           & 5                & 2                     & Llega Cliente 5 - Espera en cola                                  \\
19:20           & 3                & 1                     & Sale Cliente 3 - Heladero 2 toma un descanso                      \\
20:00           & -                & -                     & Fin periodo de observacion                                        \\ \hline
\end{tabular}
\caption{Mi Hoja de Registros}
\end{table}

\begin{parts}
\part[5] ?`Cu\'al es el tiempo promedio de los clientes en el sistema?
\fillwithlines{4 in}
\newpage
\part[10] ?`Cu\'al es n\'umero promedio de clientes en el sistema??`Cu\'al es el n\'umero promedio de clientes en cola?
\fillwithlines{4 in}
\part[5] ?`Cu\'al es la utilizaci\'on promedio de los servidores?
\fillwithlines{2 in}
\end{parts}

\newpage
\section*{Likelihood}
%\iffalse
\question Se le pide determinar el estimador de ``maximum likelihood'' que caracteriza el par\'ametro \'optimo de una Bernulli . La funci\'on de densidad se presenta a continuaci\'on.

\begin{equation}
f(x,p)=p^x(1-p)^{1-x} ~~,\text{for} ~x \in \{0,1\}  
\end{equation}

\begin{parts}
\part[15] Determine el par\'ametro $\hat{p}$.
\fillwithlines{5 in}
\end{parts}
%\fi
\newpage
\section*{Muestreo}
\question Se le entrega la siguiente funci\'on:

\begin{equation}
f_X(x) = \left\{\begin{matrix}
1/c & , 1\leq x < 2 \\ 
 1 & ~~, 2 \leq x < 2.5 \\
0&~~~~~, \text{de otro modo} \\
\end{matrix}\right.
\end{equation}

\begin{parts}
\part[3] Encuentre el valor de $c$ que hace esta funci\'on una funci\'on de densidad v\'alida.
\fillwithlines{3.2 in}
\part[6] Se le entregan los n\'umeros aleatorios $U_1 = 0.85$, $U_2 = 0.32$, y $U_3=0.78$. Genere tres muestras provenientes de la distribuci\'on por medio del m\'etodo de la transformada inversa.
\fillwithlines{3.2 in}
\newpage
\part[6] Intente generar 3 muestras por medio del m\'etodo de aceptaci\'on y rechazo. Defina la funci\'on $t(x)$ apropiada (lo mas f\'acil es definir una constante) y utilize $U_1 = 0.85$, $U_2 = 0.32$, y $U_3=0.78$ para muestrear desde $r(x)$. Finalmente, ocupe $U_1 = 0.43$, $U_2 = 0.72$, y $U_3=0.98$ para decidir si acepta o rechaza la muestra generada.
\fillwithlines{4 in}
\end{parts}
\end{questions}
\end{document}